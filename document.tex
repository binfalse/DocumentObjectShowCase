\documentclass[a4paper,10pt]{scrartcl}


\usepackage[utf8]{inputenc}


\usepackage{xcolor}
\definecolor{myblue}{cmyk}{1.00, 0.75, 0.00, 0.00}

\usepackage{tikz}
\usetikzlibrary{shapes,arrows,trees,decorations.pathmorphing,backgrounds,fit,shapes,arrows,chains, decorations.markings,shapes.arrows}
% \usepackage[a4paper]{geometry}

\renewcommand{\familydefault}{\sfdefault}
\usepackage[labelfont=bf]{caption}
\newcommand{\HRule}{\rule{\linewidth}{0.5mm}}
\setlength{\parindent}{0cm}

\def\MYTITLE{How I created a fully featured COMBINE archive}
\title{\MYTITLE}
\author{Martin Scharm}


\clubpenalty = 10000
\widowpenalty = 10000
\displaywidowpenalty = 10000

\usepackage{hyperref}
\hypersetup{
    bookmarks=true,
    unicode=false,
    pdftoolbar=true,
    pdfmenubar=true,
    pdffitwindow=false,
    pdfstartview={FitH},
    pdftitle={\MYTITLE},
    pdfauthor={martin scharm},
    pdfsubject={\MYTITLE},
    pdfcreator={martin scharm},
    pdfnewwindow=true,
    colorlinks=true,
    linkcolor=myblue,
    citecolor=myblue,
    filecolor=magenta,
    urlcolor=myblue
}

\bibliographystyle{alpha}

\begin{document}
\maketitle
% \thispagestyle{empty}
% \HRule \\[0.4cm]
% {\large Techreport, Tutorial, and Demo\\[.5em]}
% {\LARGE  \textbf{How I created a fully featured COMBINE archive}\\[.5em]}
% \HRule \\[1cm]
% 
% \begin{center}
% \Large
% Martin Scharm\\
% an probably others\\[8cm]\vfill
% \today\\
% \vfill
% \end{center}

\begin{abstract}
This is a small report on how I created the fully featured COMBINE archive
\href{https://github.com/SemsProject/CombineArchiveShowCase}{github.com/SemsProject/CombineArchiveShowCase}
to demonstrate the power of the COMBINE archive approach.
\end{abstract}

\section{Introduction}
This is a small report on how I created the fully featured COMBINE archive
\href{https://github.com/SemsProject/CombineArchiveShowCase}{github.com/SemsProject/CombineArchiveShowCase}
to demonstrate the power of the COMBINE archive approach.



\section{Methods}

\subsection{Deciding for a Simulation Study}
criteria?
what kind of study is that?


\subsection{Creating an initial COMBINE archive}
m2cat
clones from pmr2
already adds meta
opens the archive in webcat

\subsection{Extending the COMBINE archive}
\subsubsection{Retrieving Data and Information from other Services}
seek for more information on that model
and add that to the archive

\subsubsection{Generating more Data}
creating simulation description
simulating it
storing results

generating sbgn compliant figure

\section{Results}

\subsection{The COMBINE archive}
at 
\href{https://github.com/SemsProject/CombineArchiveShowCase}{github.com/SemsProject/CombineArchiveShowCase}

\subsection{The Effort}




\section{Discussion}

why is that useful?
what's the advantage?
space for improvements?

\section{Acknowledgements}

\end{document}
