
\section{Introduction}
\label{sec:intro}

The steadily increasing size and complexity of models and derived data poses the challenge of sharing reproducible results~\cite{scharm2014}.
Unfortunately, problems with replication and reproducibility are quite common~\cite{Prinz2011,repmicro,drugdevprob}.
% In times when collaborations over a distance of thousands of kilometres are mundane, 
Therefore, researchers became aware of the importance of reproducibility~\cite{Sandve2013} and
there is an increasing demand for means to transfer simulation studies, ensuring reproducibility~\cite{workflowobjects,researchobjects,Bergmann2014}.
% 
% Indeed, reproducibility is an essential challenge in systems biology~\cite{Mesirov2010}.
% The lack of reproducibility and the 
% 
% % In 2011 the pharmaceutical company Bayer tried to replicate published studies and failed in 64\% percent~\cite{Prinz2011}.
% 
Several projects and initiatives already deal with the problem of reproducibility, such as the Reproducibility Initiative\footnote{\href{http://reproducibilityinitiative.org/}{reproducibilityinitiative.org}} or FAIRDOM\footnote{\href{http://fair-dom.org/}{fair-dom.org}}.

Within the past decade the Computational Modeling in Biology Network (COMBINE) developed several standard formats to encode the different aspects of a simulation study: SBML~\cite{Hucka2003} and CellML~\cite{Cuellar2003a} to encode the biological networks; SBGN~\cite{sbgn} to encode their visual representation; \sedml~\cite{Waltemath2011} to encode the simulation recipes; NuML\footnote{\href{https://github.com/numl/numl}{github.com/numl/numl}} and SBRML~\cite{Dada2010} to encode numerical data and simulation results.
These markup languages allow for encoding single parts of a simulation study in an exchangeable format.
However, a simulation study may consist of multiple files, and the very model might have been decomposed into various modules.
\comment{Dagmar}{It would be nice to have a concrete example here, eg. the simulation study by XYZ has the following files (and describe in detail) - this would be motivating the work a lot better than simply the following sentence.}
Thus, it was still challenging to transfer reproducible research results.
% Transferring research results was still a challenge.

To close this gap, the COMBINE community developed the COMBINE archive:
A single file that aggregates all the information necessary for a modelling and simulation experiment in biology~\cite{Bergmann2014}.
The skeleton of a COMBINE archive consists of a manifest and a meta data file, specified by the Open Modeling EXchange format (OMEX).
When bundled in a COMBINE archive, essentially a .zip container, simulation experiments can be encoded in a reproducible manner. % in COMBINE archives.

To demonstrate the capabilities of a COMBINE archive I created a demo archive and made it available from our GitHub project\footnote{\href{https://github.com/SemsProject/CombineArchiveShowCase}{github.com/SemsProject/CombineArchiveShowCase}}. 
In the following I describe what I did to build that archive.
\comment{Dagmar}{Would it be useful to name and shame provenance here?}
