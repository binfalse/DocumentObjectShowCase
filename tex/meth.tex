
\section{Methods}

The development of the fully featured COMBINE archive can be divided into three major parts.
I first decided for a simulation study to encode in an archive, then I created an initial archive which was then enriched with information I found on the internet and data that I generated on my own.
These steps are described in the following.

\subsection{Deciding for a Simulation Study}

I developed 3 criteria to decide for a simulation study:

\paragraph{Criteria 1: Open Access.}
As I want to share the demo archive openly, I had to decide for an experiment that offers as much open data as possible.
Obviously and unfortunately, the publication, as an essential document for the documentation of experiments, is the bottleneck.
Therefore, I needed to find a simulation experiment which was published using an open access license.

\paragraph{Criteria 2: Much data already available in standard formats.}
As I did not want to encoded the model myself, I restricted my search to studies which are already available as SBML and CellML models.
Ideally, these models are already curated, which increases my trust in the encoding.

\paragraph{Criteria 3: I have ideally already dealt with that publication.}
As I will need to work with the study I need to understand it.
It usually takes a lot of time to dig into a new field, so I preferred studies that I already investigated.
However, this was just a soft criteria to decrease my workload.


\paragraph{Searching for a matching study.}
Searching for a study was harder than I thought.
I failed to encoded my search criteria for the search engines of current databases.
I ended up asking a common search engine to look for names of open access journals at the websites of the databases and eventually \texttt{site:models.cellml.org "Molecular Systems Biology"}\footnote{\href{https://duckduckgo.com/?q=site\%3Amodels.cellml.org+\%22Molecular+Systems+Biology\%22}{duckduckgo.com/?q=site\%3Amodels.cellml.org+\%22Molecular+Systems+Biology\%22}} resulted in a model that is available from the CellML model repository (Calzone, Thieffry, Tyson, Novak, 2007\footnote{\href{http://models.cellml.org/exposure/1a3f36d015121d5596565fe7d9afb332}{models.cellml.org/exposure/1a3f36d015121d5596565fe7d9afb332}}) \cite{cellmlrepo} and from the Biomodels Database (BIOMD0000000144\footnote{\href{http://www.ebi.ac.uk/biomodels-main/BIOMD0000000144}{www.ebi.ac.uk/biomodels-main/BIOMD0000000144}}) \cite{biomodels}.


\paragraph{The final study.}
The study I chose was published by Calzone \emph{et.~al.} in Molecular systems biology. They propose a dynamical model for the molecular events underlying rapid, synchronous, syncytial nuclear division cycles in \textit{Drosophila} embryos \cite{Calzone2007}.
In an earlier study dealing with the cell cycle I already touched that publication, so it was the perfect study for the demo archive project.




\subsection{Creating an initial COMBINE archive}
m2cat
clones from pmr2
already adds meta
opens the archive in webcat

\subsection{Extending the COMBINE archive}
\subsubsection{Retrieving Data and Information from other Services}
seek for more information on that model
and add that to the archive

\subsubsection{Generating more Data}
creating simulation description
simulating it
storing results

generating sbgn compliant figure



